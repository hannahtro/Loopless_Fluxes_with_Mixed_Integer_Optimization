
\subsection*{Preliminaries and Notation}
\todo[inline]{use variables consistently; which words to write in italic
}

\begin{enumerate}
    % \item The set $\mathbb{R}$ denotes the set of real numbers and the set $\mathbb{Z}$ denotes the set of whole numbers. If scalar $\alpha$ is in the set of real numbers, we write $\alpha \in \mathbb R$. \todo[inline]{remove?} %(instead of $\mathbb R ^1$) 
    
    \item A vector $\bold{v} = (v_1, ..., v_n) \in \mathbb{R}^n$ is identified as a column vector and printed in bold. $\bold v^\intercal$ transposes $\bold v$ into a row vector. The inner product of two vectors $\bold v, \bold w \in \mathbb{R}^n$ is $\bold v^\intercal \bold w = \sum_{i=1}^n v_i w_i$. $\bold v \leq \bold w$ denotes element-wise inequality. $\ln (\bold v)$ denotes the element-wise natural logarithm. 
    The 1-vector is written as $\bold 1 := (1, 1, ..., 1) \in \mathbb{R}^n$ the 0-vector as $\bold 0 := (0, 0, ..., 0) \in \mathbb{R}^n$. The \textit{support} of $\bold{v}$ is the set of indices $i$ with $v_i \neq 0$ and is denoted by $\text{supp}(\bold v)$. $\text{sign}(\bold v)$ is the element-wise sign function applied.
    The concatenation of two vectors $\bold x$, $\bold y$ is denoted as $(\bold x, \bold y)$.

    \item The entry in row $i$ and column $j$ of a matrix $ \bold A \in \mathbb{R}^{m \times n}$ is denoted as $a_{i,j}$. The $i$-th row is $\bold a_{i,*}$ and the $j$-th column $\bold a_{*,j}$. \todo[inline]{update in entire section, often one index used for row of matrix} The zero matrix is denoted by $\bold 0_{m,n}$ with $m$ rows and $n$ columns. $\text{diag}(\bold v)$ is the quadratic matrix with $\bold v$ on the diagonal and 0 for all other entries.
    
    \item The \textit{rank} of the matrix $\bold A \in \mathbb{R}^{m \times n}$ corresponds to the number of linearly independent columns of $\bold A$ and is denoted as $\text{rank}(\bold A)$. $\bold A$ is said to be of \textit{full rank} if $\text{rank}(\bold A) = \min \{ m, n\}$. The number of linearly independent rows and linearly independent columns are always equal.
    \todo[inline]{sufficient to say rank(A) is rank of matrix?}
    
    \item Let $\bold A \in \mathbb{R}^{n \times n}$ be an invertible matrix: $ \bold A \bold A^{-1} = \bold I$, where $\bold I$ is the identity matrix. The matrix $ \bold A^{-1}$ is the inverse matrix of $ \bold A$. %\todo[inline]{when is matrix invertible}
    $\bold I_n$ is the $n \times n$ identity matrix.

    \item A \textit{linear combination} is defined as $\sum_{i=1}^{m} \lambda_i \bold x_i = 
    \lambda_i x_1 + ... + \lambda_m x_m$, where $\lambda_i \in \mathbb{R}$ and $x_i \in \mathbb{R}^n$.
    A line going through a point $\bold x$ generated by $\bold r \in \mathbb{R}^n$ is the set $\{\bold x + \lambda \bold r | \lambda \in \mathbb{R}\}$. A \textit{line segment} is a subset of a line defined on the interval between $l \in \mathbb{R}$ and $u \in \mathbb{R}$: $\{\bold x + \lambda \bold r | \lambda \in \mathbb{R}, \, \lambda \in [l, u ]\}$.

    \item A \textit{basis} $B$ of a vector space $V$ is a set of vectors $(\bold v_1, \bold v_2, ..., \bold v_n)$ that are linearly independent and every $\bold v \in V$ can be written as a linear combination of vectors in $B$.

    \item The \textit{nullspace} of a matrix $\bold A \in \mathbb{R}^{m \times n}$ is defined as $\text{null}(\bold A):=\{\bold x \in \mathbb{R}^n: \bold A \bold x = \bold 0\}$. 
    
    \item A set $C \subseteq \mathbb{R}^n$ is \textit{convex} if for any two points $\bold x, \bold y \in C$ the line segment between $\bold x$ and $\bold y$ is in $C$. %every convex combination for any two points $x,y \in X$ is contained in $C$.
    The \textit{convex hull} of a set $X$ is the smallest convex set that contains all points in $X$ and denoted as $\text{conv}(X)$. It is a set of convex combinations such that all points $x_i$ in $X$ can be represented, where a \textit{convex combination} is a linear combination with $\lambda_i \geq 0$ and $\sum_{i=1}^m \lambda_i = 1$.
    \todo[inline]{D: should be proper definition?}

    \item The set $\{\bold x \in \mathbb{R}^n | \boldsymbol \alpha^\intercal \bold x = \boldsymbol \beta \}$ is a \textit{hyperplane}, where $\boldsymbol \alpha \in \mathbb{R}^n$ and $\boldsymbol \beta \in \mathbb{R}$. The set $\{\bold x \in \mathbb{R}^n | \boldsymbol \alpha^\intercal x \leq \boldsymbol \beta \}$ is a \textit{half-space}. %\cite{understanding_lp}
    A \textit{polyhedron} $P = \{ \bold x \in \mathbb{R}^n | \bold A \bold x \leq \bold b\}$ is the intersection of a finite number of half-spaces, where $\bold A \in \mathbb{R}^{m \times n}$ and $\bold b \in \mathbb{R}^m$. Hyperplanes, half-spaces and polyhedra are convex. A point $\bold x \in P$ is an \textit{extreme point} or \textit{vertex} if it cannot be represented as a convex combination of any set of other points in $P$. %\todo[inline]{closed halfspace needed?} 
    A \textit{polytope} is a bounded polyhedron if there exists a large enough ball in which it can be placed \cite{understanding_lp}.
    % can be written as the convex hull of the extreme points of $P$. 
    % \todo[inline]{verify}
    % \todo[inline]{add reference}

    \item A set $C \subseteq \mathbb{R}^n$ is a \textit{cone} if $\lambda \bold x \in C$ for any $\bold x \in C$ and $\lambda \geq 0$. A \textit{conic combination} is a linear combination with $\lambda_i \geq 0$. %A cone is a \textit{convex cone} if it contains the conic combinations of all $x_i \in C$. 
    $C$ is \textit{pointed} if it contains no line and the extreme point is called $apex$. A nonzero vector $\bold r \in \mathbb{R}^n$ is a \textit{ray} of $C$ if $\{\bold x + \lambda \bold r | \lambda \geq 0 \} \in C$ for any $\bold x \in C$. Ray $\bold r$ is an \textit{extreme ray} if it cannot be represented by a conic combination of other rays in $C$. 
    A cone is $polyhedral$ if the number of extreme rays is finite. A cone is \textit{simplicial} if it has $n$ extreme rays \cite{bienstock_outer_product_free_sets}. 
    The \textit{conic relaxation} of an extreme point $\bold x$ of a polyhedron $P$ is a cone with apex $\bold x$, and the extreme rays are the half-spaces of $P$ intersecting at $\bold x$.
    
    % \unsure[inline]{how to define ray, as r as lambda r or as x + lambda r}
    \item A set $S \subseteq \mathbb{R}^n$ is \textit{open} if for any point $\bold x \in S$ there exists an $\epsilon > 0$ such that the ball centered at $\bold x$ with radius $\epsilon$ is contained in $S$. A set is \textit{closed} if its complement is open. 
    % A set is \textit{closed} if it contains all its boundary points. 
    In particular, the set of whole numbers $\mathbb{Z} \subseteq \mathbb{R}$ is closed.
    % \item Let $S \in \mathbb{R}^n$ be a nonempty set and $\bold x$ a point in $S$. $\bold x$ is in the \textit{interior} of $S$, denoted as $\text{int}(S)$, if there exists an $\epsilon > 0$ such that any point in the ball centered at $\bold x$ with radius $\epsilon$ is contained in $S$. 
    % $\bold x$ is on the \textit{boundary} of $S$, denoted as $\text{bd}(S)$, if it is not in $\text{int}(S)$. 
    % $S$ is \textit{closed} if the boundary of $S$ is contained in $S$, $S$ is said to be \textit{open} otherwise. %https://wiki.math.ntnu.no/linearmethods/basicspaces/openandclosed
    % % \todo[inline]{does ball have to be open? subset of euclidean space.\\MB: you can specify if you consider balls as closed or open by default. People use open ball and closed ball explicitly when it matters.}
    % This also holds if the boundary is the empty set and therefore $\mathbb{Z}$ is closed.
    % \todo[inline]{should be definition for open, and closed is complement of open set}
\end{enumerate}