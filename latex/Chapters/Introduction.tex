\thispagestyle{plain}
\chapter*{Introduction}
\addcontentsline{toc}{chapter}{Introduction}  

In systems biology, mathematical modeling is used to predict the behavior of biological systems \cite{intro_computational_systems_biology}. A fundamental system is the cell and its metabolism.
Due to advances in sequencing and gene annotation technology, detailed information about the metabolism of numerous different organisms is available \cite{palsson_systems_biology}. Genome-scale metabolic networks model the interactions between entities in the cell such as small molecules (metabolites) and enzymes, by making use of these annotated genes \cite{intro_computational_systems_biology}.\\
A commonly used method to study metabolic networks is a constraint-based approach, which assumes a steady state of metabolites within the cell, ensuring a mass balance between the production and the consumption of each metabolite \cite{palsson_systems_biology}. The steady-state assumption, together with flux bounds on each reaction, defines a system of linear constraints. A solution corresponds to a flux distribution, which is the rate of each reaction, that satisfies the constraints \cite{intro_computational_systems_biology}. \\ %flow of metabolites through a metabolic network using mathematical optimization
A basic method to analyze the flux distribution in a metabolic network is flux balance analysis (FBA) \cite{FBA}. In FBA, a linear program is solved where we optimize a biological objective function, such as maximizing growth. As cells evolve under selection pressure, it is plausible that the behavior is in line with optimizing growth\todo[inline]{was this meant??} and using mathematical optimization to study the flow of metabolites through a metabolic network is reasonable \cite{palsson_systems_biology}. With FBA, it is possible to predict cellular behavior in accordance with observations in experiments~\cite{FBA}. \\
%% MOTIVATION AND CONTEXT 1-2 pages
%% get attention and introduce topic
%% background information
% \todo[inline]{question addressed system biology}
% System biologists model large biological systems with the help of mathematical approaches to predict the behaviour of the system \cite{intro_computational_systems_biology}. 
% \todo[inline]{goal of modelling}
% \todo[inline]{question addressed with studying metabolic models} 
% \todo[inline]{metabolic models}
% \todo[inline]{availability of GEMs}
% Due to advances in technology, 
% quantitative data on biological systems.... detailed information of the cell ...
% We have access to the composition of ... give rise to reconstructed biochemical reaction networks \cite{palsson_systems_biology}.
% \todo[inline]{metabolic network}
% \todo[inline]{FBA} 
% Flux balance analysis is an approach for analyzing the flow of metabolites through a metabolic network using mathematical optimization \cite{FBA}.
% \todo[inline]{success and use of FBA}
% \todo[inline]{problems with FBA}
However, solutions of FBA often violate the loop law, which means that the solution contains internal cycles which is biologically implausible\todo[inline]{was this meant??}. Loopless FBA (ll-FBA) is an extension to FBA which includes thermodynamic information \cite{elimination_infeasible_loops}. A solution to ll-FBA does not contain internal cycles, but ll-FBA is computationally harder as we need to solve a disjunctive program, which is usually reformulated as a mixed-integer program (MIP). 
In this thesis, we look at different reformulations of ll-FBA and compare the performance of different solution approaches. %\todo[inline]{MIP technique correct for all/most approaches?}

%% thesis statement: topic, question being addressed, summarize resulst and significance
%% STRATEGY 1 page: what am I trying to do

%% OUTLINE 1/2 page
%% and outlines structure to come
% \todo[inline]{overview of thesis}

In \cref{chap:optimization}, we will give a short introduction to mathematical optimization with a focus on the aspects relevant to this thesis including the problem definitions and solving strategies for linear programming, mixed-integer programming and disjunctive programming. We will present the biological context in more detail and present the relevant problems in detail in \cref{chap:metabolic_networks}. 
We consider several reformulations and solution strategies in \cref{chap:methods}, including a combinatorial Benders' approach, and we establish how to apply intersection cuts to ll-FBA. In \cref{chap:results}, we show and discuss the computational results. 
% \todo[inline]{FBA extensions}

\thispagestyle{plain}
\section*{Preliminaries and Notation}
% \todo[inline]{use variables consistently; which words to write in italic}

\begin{enumerate}
    % \item The set $\mathbb{R}$ denotes the set of real numbers and the set $\mathbb{Z}$ denotes the set of whole numbers. If scalar $\alpha$ is in the set of real numbers, we write $\alpha \in \mathbb R$. \todo[inline]{remove?} %(instead of $\mathbb R ^1$) 
    
    \item A vector $\bold{v} = (v_1, ..., v_n) \in \mathbb{R}^n$ is identified as a column vector and printed in bold. The transpose of $\mathbf v$ into a row vector is written as $\mathbf v^\intercal$. The inner product of two vectors $\mathbf v, \mathbf w \in \mathbb{R}^n$ is $\mathbf v^\intercal \mathbf w = \sum_{i=1}^n v_i w_i$. With $\mathbf v \leq \mathbf w$ we denote element-wise inequality. With $\ln (\mathbf v)$ we denote the element-wise natural logarithm. 
    The 1-vector is written as $\mathbf 1 := (1, 1, ..., 1) \in \mathbb{R}^n$ the 0-vector as $\mathbf 0 := (0, 0, ..., 0) \in \mathbb{R}^n$. The \textit{support} of $\bold{v}$ is the set of indices $i$ with $v_i \neq 0$ and is denoted by $\text{supp}(\mathbf v)$. With $\text{sign}(\mathbf v)$ the element-wise sign function is applied to $\bold v$.
    The concatenation of two vectors $\mathbf x$, $\mathbf y$ is denoted by $(\mathbf x, \mathbf y)$.

    \item The entry in row $i$ and column $j$ of a matrix $ \mathbf A \in \mathbb{R}^{m \times n}$ is denoted by $a_{i,j}$. The $i$-th row is $\boldsymbol a_{i,*}$ and the $j$-th column $\boldsymbol a_{*,j}$. %\todo[inline]{update in entire section, often one index used for row of matrix} 
    The zero matrix is denoted by $\mathbf 0_{m,n}$ with $m$ rows and $n$ columns. With $\text{diag}(\mathbf v)$ we denote the quadratic matrix with $\mathbf v$ on the diagonal and 0 for all other entries.
    
    \item The \textit{rank} of the matrix $\mathbf A \in \mathbb{R}^{m \times n}$ corresponds to the number of linearly independent columns of $\mathbf A$ and is denoted by $\text{rank}(\mathbf A)$. The matrix $\mathbf A$ is said to be of \textit{full rank} if $\text{rank}(\mathbf A) = \min \{ m, n\}$. The number of linearly independent rows and linearly independent columns is always equal.
    % \todo[inline]{sufficient to say rank(A) is rank of matrix?}
    
    \item Let $\mathbf A \in \mathbb{R}^{n \times n}$ be an invertible matrix: $ \mathbf A \mathbf A^{-1} = \mathbf I$, where $\mathbf I$ is the identity matrix. The matrix $ \mathbf A^{-1}$ is the inverse matrix of $ \mathbf A$. %\todo[inline]{when is matrix invertible}
    $\mathbf I_n$ is the $n \times n$ identity matrix.

    \item A \textit{linear combination} is defined as $\sum_{i=1}^{m} \lambda_i \mathbf x_i = 
    \lambda_i x_1 + ... + \lambda_m x_m$, where $\lambda_i \in \mathbb{R}$ and $x_i \in \mathbb{R}^n$.
    A line going through a point $\mathbf x$ generated by $\mathbf r \in \mathbb{R}^n$ is the set $\{\mathbf x + \lambda \mathbf r | \lambda \in \mathbb{R}\}$. A \textit{line segment} is a subset of a line defined on the interval between $l \in \mathbb{R}$ and $u \in \mathbb{R}$: $\{\mathbf x + \lambda \mathbf r | \lambda \in \mathbb{R}, \, \lambda \in [l, u ]\}$.

    \item A \textit{basis} $B$ of a vector space $V$ is a set of vectors $(\mathbf v_1, \mathbf v_2, ..., \mathbf v_n)$ that are linearly independent and every $\mathbf v \in V$ can be written as a linear combination of vectors in $B$.

    \item The \textit{nullspace} of a matrix $\mathbf A \in \mathbb{R}^{m \times n}$ is defined as $\text{null}(\mathbf A):=\{\mathbf x \in \mathbb{R}^n: \mathbf A \mathbf x = ~\mathbf0\}$. 
    \vspace*{-\baselineskip}
    
    \newpage
    \item A set $C \subseteq \mathbb{R}^n$ is \textit{convex} if for any two points $\mathbf x, \mathbf y \in C$ the line segment between $\mathbf x$ and $\mathbf y$ is in $C$. %every convex combination for any two points $x,y \in X$ is contained in $C$.
    The \textit{convex hull} of a set $X$ is the smallest convex set that contains all points in $X$ and is denoted by $\text{conv}(X)$. It is a set of convex combinations such that all points $x_i$ in $X$ can be represented, where a \textit{convex combination} is a linear combination with $\lambda_i \geq 0$ and $\sum_{i=1}^m \lambda_i = 1$.
    % \todo[inline]{D: should be proper definition?}

    \item The set $\{\mathbf x \in \mathbb{R}^n | \boldsymbol \alpha^\intercal \mathbf x = \boldsymbol \beta \}$ is a \textit{hyperplane}, where $\boldsymbol \alpha \in \mathbb{R}^n$ and $\boldsymbol \beta \in \mathbb{R}$. The set $\{\mathbf x \in \mathbb{R}^n | \boldsymbol \alpha^\intercal x \leq \boldsymbol \beta \}$ is a \textit{half-space}. %\cite{understanding_lp}
    A \textit{polyhedron} $P = \{ \mathbf x \in \mathbb{R}^n | \mathbf A \mathbf x \leq \mathbf b\}$ is the intersection of a finite number of half-spaces, where $\mathbf A \in \mathbb{R}^{m \times n}$ and $\mathbf b \in \mathbb{R}^m$. Hyperplanes, half-spaces and polyhedra are convex. A point $\mathbf x \in P$ is an \textit{extreme point} or \textit{vertex} if it cannot be represented as a convex combination of any set of other points in $P$. %\todo[inline]{closed halfspace needed?} 
    A \textit{polytope} is a bounded polyhedron if there exists a large enough ball in which it can be placed \cite{understanding_lp}.
    % can be written as the convex hull of the extreme points of $P$. 
    % \todo[inline]{verify}
    % \todo[inline]{add reference}

    \item A set $C \subseteq \mathbb{R}^n$ is a \textit{cone} if $\lambda \mathbf x \in C$ for any $\mathbf x \in C$ and $\lambda \geq 0$. A \textit{conic combination} is a linear combination with $\lambda_i \geq 0$. %A cone is a \textit{convex cone} if it contains the conic combinations of all $x_i \in C$. 
    $C$ is \textit{pointed} if it contains no line and the extreme point is called $apex$. A nonzero vector $\mathbf r \in \mathbb{R}^n$ is a \textit{ray} of $C$ if $\{\mathbf x + \lambda \mathbf r | \lambda \geq 0 \} \in C$ for any $\mathbf x \in C$. Ray $\mathbf r$ is an \textit{extreme ray} if it cannot be represented by a conic combination of other rays in $C$. 
    A cone is $polyhedral$ if the number of extreme rays is finite. A cone is \textit{simplicial} if it has $n$ extreme rays \cite{bienstock_outer_product_free_sets}. 
    The \textit{conic relaxation} of an extreme point $\mathbf x$ of a polyhedron $P$ is a cone with apex $\mathbf x$, and the extreme rays are the half-spaces of $P$ intersecting at $\mathbf x$.
    
    % \unsure[inline]{how to define ray, as r as lambda r or as x + lambda r}
    \item A set $S \subseteq \mathbb{R}^n$ is \textit{open} if for any point $\mathbf x \in S$ there exists an $\epsilon > 0$ such that the ball centered at $\mathbf x$ with radius $\epsilon$ is contained in $S$. A set is \textit{closed} if its complement is open. 
    % A set is \textit{closed} if it contains all its boundary points. 
    In particular, the set of whole numbers $\mathbb{Z} \subseteq \mathbb{R}$ is closed.
    % \item Let $S \in \mathbb{R}^n$ be a nonempty set and $\mathbf x$ a point in $S$. $\mathbf x$ is in the \textit{interior} of $S$, denoted by $\text{int}(S)$, if there exists an $\epsilon > 0$ such that any point in the ball centered at $\mathbf x$ with radius $\epsilon$ is contained in $S$. 
    % $\mathbf x$ is on the \textit{boundary} of $S$, denoted by $\text{bd}(S)$, if it is not in $\text{int}(S)$. 
    % $S$ is \textit{closed} if the boundary of $S$ is contained in $S$, $S$ is said to be \textit{open} otherwise. %https://wiki.math.ntnu.no/linearmethods/basicspaces/openandclosed
    % % \todo[inline]{does ball have to be open? subset of euclidean space.\\MB: you can specify if you consider balls as closed or open by default. People use open ball and closed ball explicitly when it matters.}
    % This also holds if the boundary is the empty set and therefore $\mathbb{Z}$ is closed.
    % \todo[inline]{should be definition for open, and closed is complement of open set}
\end{enumerate}
\thispagestyle{plain}

