\usepackage[utf8]{inputenc} % Kodierung
\usepackage[T1]{fontenc} % Explizite Nennung des Fonts
%\usepackage{tgschola}
%\usepackage{utopia}
\usepackage{helvet}

\usepackage{xargs}                      % Use more than one optional parameter in a new commands
\usepackage[pdftex,dvipsnames]{xcolor}  % Coloured text etc.

% \usepackage[colorinlistoftodos,prependcaption,textsize=small]{todonotes}
\usepackage[disable]{todonotes}
\newcommandx{\unsure}[2][1=]{\todo[linecolor=red,backgroundcolor=red!25,bordercolor=red,#1]{#2}}
% \newcommandx{\change}[2][1=]{\todo[linecolor=blue,backgroundcolor=blue!25,bordercolor=blue,#1]{#2}}
% \newcommandx{\info}[2][1=]{\todo[linecolor=OliveGreen,backgroundcolor=OliveGreen!25,bordercolor=OliveGreen,#1]{#2}}
% \newcommandx{\improvement}[2][1=]{\todo[linecolor=Plum,backgroundcolor=Plum!25,bordercolor=Plum,#1]{#2}}
% \newcommandx{\thiswillnotshow}[2][1=]{\todo[disable,#1]{#2}}

%\renewcommand\familydefault{\sfdefault} 
\usepackage[english,ngerman]{babel} % Sprache %ngermen

\usepackage{datetime}
\newdateformat{myformat}{\THEDAY{. }\monthname[\THEMONTH] \THEYEAR}

\usepackage{graphicx} % immer benötigt für das Einbinden von Graphiken
\usepackage{caption} % Beschriftung von Bildern 
\captionsetup[sub]{font=small, labelfont=normalsize}
\usepackage[font=small]{subcaption}

\usepackage{blindtext} % Wenn man das Layout prüfen will, kann hier mit \blindtext Text eingefügt werden.

\usepackage{parskip} % Für den Abstand zwischen 2 Absätzen.
%\setlength{\parskip}{12pt plus80pt minus10pt} % Genaue Einstellung von parskip

\usepackage{lmodern}

%Literaturverzeichnis
\usepackage[style=trad-abbrv,backend=biber, backref=false]{biblatex}
 % Biber backend für Literaturverzeichnis
% \addbibresource{bibliography.bib} % Einbinden der Literatur.
%\DeclareLanguageMapping{english}{english-apa} % Anpassen Spracheinstellungen im Literaturverzeichnis.

\usepackage[activate={true,nocompatibility},
	final,
	tracking=true,
	kerning=true,
	expansion=true,
	spacing=true,
	factor=1050,
	stretch=25,
	shrink=10]{microtype} % Für die Feineinstellung der Zeichensetzung.

\usepackage[right=3 cm, left=3cm, top=2.5 cm, bottom=3 cm]{geometry} % Seitenränder
  \linespread{1.25}

\usepackage{appendix}
\usepackage{setspace}
\usepackage{fancyhdr} % Für schönere Kopf-/Fußzeilen und Fußnoten.
\pagestyle{fancy}
\setlength{\headheight}{14.49998pt}

\usepackage[rflt]{floatflt} % Bild in Text
\usepackage[]{graphicx} %Bilder %change draft to final
% \usepackage{subcaption} % Bildunterschrift
\usepackage{booktabs} % für Tabellen
\usepackage{pbox} % für Boxen
\usepackage{tabulary} %für Tabellen
\usepackage{comment} % für Kommentare 
\usepackage{amssymb}
% \usepackage[fleqn]{amsmath}
\usepackage{amsmath} 
% \usepackage{upgreek} 
% \let\mu\upmu 

\usepackage{mathtools}
\usepackage{mathrsfs}
\usepackage[nocomma,short]{optidef}
\usepackage{gensymb}

\usepackage{rotating}
\usepackage{makecell}
\usepackage{tabto}
\usepackage{beramono}

% \usepackage{minted}

\usepackage[nocomma,short]{optidef}
\newcommand*{\matr}[1]{\mathbf it{#1}}
\newcommand*{\tran}{^{\mkern-1.5mu\mathsf{T}}}

% \usepackage{mdframed}
\newtheorem{theorem}{Theorem}
\newtheorem{lemma}{Lemma}
\newtheorem{problem}{Problem}

\usepackage{verbatim} % Text ohne Formatierung ausgeben
\usepackage{fancyvrb} % Verbatim konfigurieren
\usepackage{csquotes} % Für ordentlichen Anführungszeichen

%\usepackage{floatpag} %suppress pagenumber
%\captionsetup[sub]{font=small,labelfont={bf,sf}}
% placing figure at top of page
\makeatletter
\setlength{\@fptop}{0pt}
\makeatother

\newcommand{\rom}[1]{\uppercase\expandafter{\romannumeral #1\relax}}

\usepackage{hyperref}
% \usepackage[hidelinks]{hyperref} % Klickbare aber nicht markierte Links im PDF
\hypersetup{colorlinks={true},linkcolor={blue},urlcolor=blue, citecolor=black, anchorcolor=blue}
\usepackage[capitalise,noabbrev,nameinlink]{cleveref} 
\usepackage{nameref}
\Crefname{problem}{Problem}{Problems}
\crefname{problem}{Problem}{Problems}
\Crefname{constraint}{Constraint}{Constraints}
\crefname{constraint}{Constraint}{Constraints}
\Crefname{table}{Table}{Tables}
\crefdefaultlabelformat{(#2#1#3)}
% \creflabelformat{problem}{#2\textup{#1}#3}
\creflabelformat{section}{#2\textup{#1}#3}
\creflabelformat{theorem}{#2\textup{#1}#3}
\creflabelformat{figure}{#2\textup{#1}#3}
\creflabelformat{algorithm}{#2\textup{#1}#3}
\creflabelformat{table}{#2\textup{#1}#3}
\creflabelformat{apptab}{#2\textup{#1}#3}
\creflabelformat{chapter}{#2\textup{#1}#3}

\counterwithout{table}{chapter}
\counterwithout{figure}{chapter}

\crefname{appfig}{Appendix Figure}{Appendix Figures}
\crefname{apptab}{Appendix Table}{Appendix Tables}
\Crefname{appfig}{Appendix Figure}{Appendix Figures}
\Crefname{apptab}{Appendix Table}{Appendix Tables}

% \creflabelformat{constraint}{#2\textup{#1}#3}
% \creflabelformat{equation}{#2\textup{#1}#3}
\crefrangelabelformat{constraint}{(#3#1#4--#5\crefstripprefix{#1}{#2}#6)}

% \usepackage{algorithmicx}
\usepackage{algorithm}% http://ctan.org/pkg/algorithms
\usepackage{algpseudocode}% http://ctan.org/pkg/algorithmicx

\usepackage{makecell}
\renewcommand\theadalign{bc}
\renewcommand\theadfont{\bfseries}

\usepackage{float}

\newcommand{\RomanNumeralCaps}[1]{\MakeUppercase{\romannumeral #1}}

\usepackage{chngcntr}
% \counterwithin{figure}{subsection}
% \counterwithin{table}{section}
% \counterwithin{equation}{section}

\usepackage[nottoc]{tocbibind}% load before tocbasic
\usepackage[titles]{tocloft}
\usepackage{tocbasic}
\usepackage{scrbase}
\renewcommand*{\tableofcontents}{\listoftoc[{\contentsname}]{toc}}
% \renewcommand*{\listoffigures}{\listoftoc[{\listfigurename}]{lof}}
% \renewcommand{\cftloftitlefont}{\hfill\large\bfseries}
\renewcommand*{\listoftables}{\listoftoc[{\listtablename}]{lot}}
\cftsetindents{figure}{1.5em}{3.5em}

% \renewcommand\listoffigures{%
%     \section{\listfigurename}% Used to be \section*{\listfigurename}
%       \@mkboth{\MakeUppercase\listfigurename}%
%               {\MakeUppercase\listfigurename}%
%     \@starttoc{lof}%
%     }

\makeatletter
\newcommand\renewlistof[3]%
   {\renewcommand#1%
      {\section*{#3}%
      %  \addcontentsline{toc}{chapter}{#3}%
       \markboth{#3}{#3}%
       \@starttoc{#2}%
      }%
   }
\makeatother
\renewlistof\listoffigures{lof}{\listfigurename}

\DeclareNewTOC[
  type=myequation,
  name=Equation,
  listname={\large List of Equations},
  tocentrynumwidth=2.3em,% like figures and tables
  tocentryindent=1.5em,% like figures and tables
]{equ}
\newcommand{\myequations}[1]{%
  \addxcontentsline{equ}{myequation}[\theequation]{#1}}
\setuptoc{equ}{leveldown,totoc}

\DeclareNewTOC[
  type=myproblem,
  name=Problem,
  listname={\Large List of Problems},
  tocentrynumwidth=2.3em,% like figures and tables
  tocentryindent=1.5em,% like figures and tables
]{pro}
\newcommand{\myproblems}[1]{%
%   \addxcontentsline{pro}{myproblem}[\theproblem]{#1}}
  \addcontentsline{pro}{myproblem}{#1}}
\setuptoc{pro}{leveldown}
\counterwithout{equation}{chapter}

% \usepackage{tocloft}
% \usepackage[titles]{tocloft}
% \newcommand{\listequationsname}{List of Equations}
% \newlistof{myequations}{equ}{\listequationsname}
% \newcommand{\myequations}[1]{%
% \addcontentsline{equ}{myequations}{\protect\numberline{\theequation}#1}\par}

% \newcommand{\listproblemname}{List of Mathematical Programs}
% \newlistof{myproblems}{pro}{\listproblemsname}
% \newcommand{\myproblems}[1]{%
% \addcontentsline{pro}{myproblems}{\protect\numberline{\theproblem}#1}\par}

\usepackage{enumitem}
\usepackage{acro}

\usepackage{changepage}

\usepackage{etoolbox}
\robustify{\label}

\usepackage{setspace}

\usepackage{pdfpages}